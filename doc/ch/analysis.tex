\chapter{Analysis}


%\section{Analyze time distribution}

In order to reduce the amount of CPU time involved in each step of the 
simulation, the best strategy is to reduce the time spent in the most time 
consuming part.

The CPU time involved in each part of the simulation may depend on various 
factors, such as the number of grid points, the number of particles or the 
boundary conditions. As an example, consider a simulation with a large number of 
grid points, with few particles---the computation of the electric field 
(\texttt{field\_E}) will dominate the simulation time, as shown in the figure 
\ref{fig:cm-big-grid}. In a case of a large number of particles and a smaller 
grid, the particle interpolation (\texttt{particle\_E}) dominates the whole 
execution as seen in the figure \ref{fig:cm-lots-particles}.
%
\begin{figure}[h]
	\centering
	\subfloat[1024 particles, 512x512 grid points]{
		\includegraphics[width=0.95\linewidth]{callmap-grid512x512-n1024.png}
		\label{fig:cm-big-grid}
	}
	\\
	\subfloat[10240 particles, 64x64 grid points]{
		\includegraphics[width=0.95\linewidth]{callmap-grid64x64-n10240.png}
		\label{fig:cm-lots-particles}
	}
	\caption{Comparison of the time spent in each function at two different 
	simulations.}
\end{figure}
%
In order to optimize the general use case, different inputs will be tested and 
the main simulation steps will be characterized. Furthermore, different 
algorithms or methods may be used to improve the speed. As an example, the LU 
algorithm is compared with the spectral method MFT.

\section{Analysis with varying inputs}

