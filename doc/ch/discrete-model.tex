\chapter{Discrete model}
\label{ch:discrete-model}

The mathematical model is discretized in algebraic operations, in order to be 
computable.

\section{Charge assignment}

At each grid point $g$ at $(x,y)$ we accumulate the charge of each particle $p$ 
as

\begin{equation}
\rho(x, y) = \sum_p W(\x_g(x, y) - \x_p) + \rho_0
\end{equation}

\section{Field equations}

The electric potential $\phi$ can be obtained from the charge density $\rho$.  
Several methods are available to solve the Poisson equation (eq.  
\ref{eq:poisson}).
%
The electric field can be obtained by finite differences as
%
\begin{align}
\E_x(x,y) &= \frac{\phi(x-1,y) - \phi(x+1,y)}{2\,\Delta x} \\
\E_y(x,y) &= \frac{\phi(x,y-1) - \phi(x,y+1)}{2\,\Delta y}
\end{align}

\paragraph{Iterative  methods} such as Jacobi, Gauss-Seidel, Successive Over 
Relaxation (SOR), Chebyshev acceleration are some of the most familiar methods 
to solve the Poisson equation.

\paragraph{Matrix methods} The equations from finite differencing the mesh are 
considered a large system of equations. We can find in this methods the Thomas 
Tridiagonal algorithm, Conjugate-Gradient, LU or Incomplete Decomposition.

\paragraph{Spectral methods} Also known as Rapid Elliptic Solvers (RES) are a 
family of methods that use the fast Fourier transform (FFT). Are know for being 
usually faster than the previous ones, with a complexity in $O(N_g \log_2 N_g)$

We will focus on the spectral methods, more specific on the Multiple Fourier 
Transform (MTF) method, as it is the main method implemented in the simulator.

\subsection{Multiple Fourier Transform (MFT)}

The general second-order PDE with constant coefficients and periodic boundary 
conditions
%
\begin{equation}
\label{eq:gen-fd}
a \frac{\partial^2 \phi}{\partial x^2}+b\frac{\partial \phi}{\partial x}+c\phi +
d \frac{\partial^2 \phi}{\partial y^2}+e\frac{\partial \phi}{\partial y}+f\phi
\end{equation}
%
can be solved by using the FFT. If we expand $\phi$ and $g$ in a finite double 
Fourier series, we obtain
%
\begin{equation}
\phi(x,y) = \sum_{k,l} \hat \phi(k, l) \exp\left({\frac{2\pi i (xk + 
yl)}{n}}\right)
\end{equation}
%
and
%
\begin{equation}
g(x,y) = \sum_{k,l} \hat g(k, l) \exp\left({\frac{2\pi i (xk + yl)}{n}}\right)
\end{equation}
%
which now can be substituted in the Eq.~\ref{eq:gen-fd}, to obtain
%
\begin{equation}
\hat \phi(k,l) = \hat G(k,l) \, \hat g(k,l),\quad 0<k,l<n-1
\end{equation}
%
with for a unit mesh
%
\begin{equation}
\begin{split}
\hat G(k,l) = \Bigg[
& 2a \left( \cos \frac{2\pi k}{n} - 1 \right) +
ib \sin \frac{2\pi k}{n} + c \,+ \\
& 2d \left( \cos \frac{2\pi l}{n} - 1 \right) +
ie \sin \frac{2\pi l}{n} + f
\Bigg]^{-1}
\end{split}
\end{equation}
%
To solve the Poisson equation, discretized as Eq. [missing], we have $a=d=1$ and 
$b=c=e=f=0$ so we can simplify $\hat G(k,l)$ as
%
\begin{equation}
\hat G(k,l) = \frac{1}{2}\left[
\cos \frac{2\pi k}{n} +
\cos \frac{2\pi l}{n} -
2 \right]^{-1}
\end{equation}
%
The steps to compute the electric field can be summarized as follows:
\begin{enumerate}
\item Compute the complex FFT $\hat g$ of $g$
\item Multiply each element of $\hat g$ by the corresponding complex coefficient 
$\hat G$, to obtain $\hat \phi$
\item Compute the inverse FFT of $\hat \phi$ to get $\phi$
\end{enumerate}
%
The complexity in the worst case is in $O(N_g \log_2 N_g)$ with the number of 
total points in the grid $N_g$.

\section{Force interpolation}
\section{Equations of motion}

