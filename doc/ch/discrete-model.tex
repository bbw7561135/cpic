\chapter{Discrete model}
\label{ch:discrete-model}

The mathematical model is discretized in algebraic operations, in order to be 
computable.

\section{Charge assignment}
At each grid point $g$ at $(x,y)$ we accumulate the charge of each particle $p$ 
in $(x_p, y_p)$ as
%
\begin{equation}%{{{
\rho(x, y) = \sum_p q\,W(x - x_p,\ y - y_p) + \rho_0
\end{equation}%}}}
%
The background charge density $\rho_0$ is used to neutralize the total charge 
when is non-zero. The weighting function $W$ determines the shape of the 
particle charge. Different schemes can be used to approximate the charge density 
from the particles. We will focus on bilinear interpolation for it's simplicity 
and low computation requirements. The corresponding weighting function can be 
written as
%
\begin{equation}%{{{
W(x, y) =
\begin{cases}
			\displaystyle\frac{(1-|x|)(1-|y|)}{\Delta x \Delta y} & \text{if}\ x \le 
			\Delta x,\text{ and } y\le \Delta y \\
			0 & \text{otherwise}
\end{cases}
\end{equation}%}}}
%
Then, a particle $p$ always affects the four enclosing grid points in the 
neighbourhood, but more complex interpolation methods may extend the update 
region even further. It may be noted that the increase in smoothing, at 
computation expense, can gain from the reduced number of particles needed to 
obtain a similar result, avoiding nonphysical effects.
%
\begin{figure}[]%{{{
\centering
\begin{tikzpicture}[
		>=latex,
		effect/.style={dashed,-{Latex[length=3mm, width=1mm]}},
		particle/.style={fill=black,radius=3pt},
	]
	\draw [step=2cm,dotted] (1,1) grid (5,5);
	\coordinate (p) at (2.5,3.2);
	\coordinate (center) at (3,3);
	\coordinate (A) at ($(center)+(-1,1)$);
	\coordinate (B) at ($(center)+(1,1)$);
	\coordinate (C) at ($(center)+(1,-1)$);
	\coordinate (D) at ($(center)+(-1,-1)$);
	\draw[effect] (p) -- (A);
	\draw[effect] (p) -- (B);
	\draw[effect] (p) -- (C);
	\draw[effect] (p) -- (D);
	\node[above left]  at (A) {$A$};
	\node[above right] at (B) {$B$};
	\node[below right] at (C) {$C$};
	\node[below left]  at (D) {$D$};
	\draw[particle] (p) circle;
	\node[left] at (p) {$p$};
\end{tikzpicture}
\hspace{0.5cm}
\begin{tikzpicture}[
		>=latex,
		box/.style={black},
		particle/.style={fill=black,radius=3pt},
		div/.style={dashed},
	]
	\draw [step=2cm,dotted] (1,1) grid (5,5);
	\coordinate (p) at (2.5,3.2);
	\coordinate (center) at (3,3);
	\coordinate (A) at ($(p)+(-1,1)$);
	\coordinate (B) at ($(p)+(1,1)$);
	\coordinate (C) at ($(p)+(1,-1)$);
	\coordinate (D) at ($(p)+(-1,-1)$);
	\draw[box] (A) -- (B) -- (C) -- (D) -- (A);
	\draw[div] ($(A)!(center)!(B)$) -- (center);
	\draw[div] ($(B)!(center)!(C)$) -- (center);
	\draw[div] ($(C)!(center)!(D)$) -- (center);
	\draw[div] ($(D)!(center)!(A)$) -- (center);
	\node at ($(center)!0.5!(A)$) {$a$};
	\node at ($(center)!0.5!(B)$) {$b$};
	\node at ($(center)!0.5!(C)$) {$c$};
	\node at ($(center)!0.5!(D)$) {$d$};
	\draw[particle] (p) circle;
\end{tikzpicture}
\hspace{0.5cm}
\begin{tikzpicture}[
		>=latex,
		box/.style={black},
		particle/.style={fill=black,radius=3pt},
		div/.style={dashed},
	]
	\draw [step=2cm,dotted] (1,1) grid (5,5);
	\coordinate (p) at (2.5,3.2);
	\coordinate (center) at (3,3);
	\coordinate (A) at ($(center)+(-1,1)$);
	\coordinate (B) at ($(center)+(1,1)$);
	\coordinate (C) at ($(center)+(1,-1)$);
	\coordinate (D) at ($(center)+(-1,-1)$);
	\draw[box] (A) -- (B) -- (C) -- (D) -- (A);
	\draw[div] ($(A)!(p)!(B)$) -- (p);
	\draw[div] ($(B)!(p)!(C)$) -- (p);
	\draw[div] ($(C)!(p)!(D)$) -- (p);
	\draw[div] ($(D)!(p)!(A)$) -- (p);
	\node at ($(p)!0.5!(A)$) {$c$};
	\node at ($(p)!0.5!(B)$) {$d$};
	\node at ($(p)!0.5!(C)$) {$a$};
	\node at ($(p)!0.5!(D)$) {$b$};
	\draw[particle] (p) circle;
\end{tikzpicture}
\caption{Interpolation of particle $p$ charge into the four grid points A to D.}
\label{fig:interpolation}
\end{figure}%}}}
%
The particle $p$ has a uniform charge area, centered at the particle position, 
with size $\Delta \x$, as shown in the figure~\ref{fig:interpolation}. Each grid 
point $A,B,C$ and $D$ receives the amount of charge weighed by the area $a,b,c$ 
and $d$. It can be observed that the area is equal to the opposite region, when 
the particle $p$ is used to divide the grid cell.
%
The particle shape can be altered later in the Fourier space, without large 
computation effort, in case the solver already computes the FFT.
%
\section{Field equations}

The electric potential $\phi$ can be obtained from the charge density $\rho$.  
Several methods are available to solve the Poisson equation 
(Eq.~\ref{eq:poisson}):

\paragraph{Iterative  methods} such as Jacobi, Gauss-Seidel, Successive Over 
Relaxation (SOR), Chebyshev acceleration are some of the most familiar methods 
to solve the Poisson equation.

\paragraph{Matrix methods} The equations from finite differencing the mesh are 
considered a large system of equations. We can find in this methods the Thomas 
Tridiagonal algorithm, Conjugate-Gradient, LU or Incomplete Decomposition.

\paragraph{Spectral methods} Also known as Rapid Elliptic Solvers (RES) are a 
family of methods that use the fast Fourier transform (FFT). Are know for being 
usually faster than the previous ones, with a complexity in $O(N_g \log_2 N_g)$

\vspace{1em}
\noindent
%
We will focus on the spectral methods, more specific on the Multiple Fourier 
Transform (MTF) method, as it is the main method implemented in the simulator, 
due to its relative simplicity and low computational complexity.

\subsection{LU decomposition}

%
% TODO: Check the error bound
For two dimensions, we can approximate the solution using the second order 
centered finite differences (with an error proportional to $\Delta x ^2 \Delta 
y^2$), as
%
\begin{equation}%{{{
\label{eq:discrete-poisson}
\frac{\phi(x-1, y) + \phi(x, y-1) - 4\phi(x,y) + \phi(x+1,y)+\phi(x,y+1)}{\Delta 
x ^2 \Delta y^2} = - \frac{\rho(x,y)}{\epsilon_0}
\end{equation}%}}}
%
which leads to a system of linear equations with $N_g$ equations.
The electric field $\E$ can then be obtained by centered first order finite 
differences in each dimension
%
\begin{equation}%{{{
\begin{split}
\label{eq:phi-to-E}
\E_x(x,y) &= \frac{\phi(x-1,y) - \phi(x+1,y)}{2\,\Delta x} \\
\E_y(x,y) &= \frac{\phi(x,y-1) - \phi(x,y+1)}{2\,\Delta y}
\end{split}
\end{equation}%}}}
%



The system of $N_g$ equations formed by the equation~\ref{eq:discrete-poisson} 
can be also written in matrix form
%
\begin{equation}
\label{eq:eq-system}
A\phi = -\frac{\Delta x ^2 \Delta y^2\,\rho}{\epsilon_0}
\end{equation}
%
The $N_g \times N_g$ coefficient matrix $A$ has non-zero coefficients only at 
$a_{ii} = 4$ and $a_{ij} = -1$ with $j \in \{i+1, i-1, i+N_x, i-N_x\} \mod N_x$, 
for all $0 \le i \le Ng$.
%
The $LU$ decomposition can be used to form two systems of equations that can be 
solved faster. If we rewrite the system of equations~\ref{eq:eq-system} as the 
usual form $Ax=b$ with
\begin{equation}
x = \phi,\quad b = -\frac{\Delta x ^2 \Delta y^2\,\rho}{\epsilon_0}
\end{equation}
%
Then we can use the decomposition $A=LU$ to form two systems of equations
%
\begin{equation}
Ux=y, \quad Ly=b
\end{equation}



\subsection{Multiple Fourier Transform (MFT)}

The general second-order PDE with constant coefficients and periodic boundary 
conditions
%
\begin{equation}%{{{
\label{eq:gen-fd}
a \frac{\partial^2 \phi}{\partial x^2}+b\frac{\partial \phi}{\partial x}+c\phi +
d \frac{\partial^2 \phi}{\partial y^2}+e\frac{\partial \phi}{\partial y}+f\phi
\end{equation}%}}}
%
can be solved by using the FFT. If we expand $\phi$ and $g$ in a finite double 
Fourier series, we obtain
%
\begin{equation}%{{{
\phi(x,y) = \sum_{k,l} \hat \phi(k, l) \exp\left({\frac{2\pi i (xk + 
yl)}{n}}\right)
\end{equation}%}}}
%
and
%
\begin{equation}%{{{
g(x,y) = \sum_{k,l} \hat g(k, l) \exp\left({\frac{2\pi i (xk + yl)}{n}}\right)
\end{equation}%}}}
%
which now can be substituted in the Eq.~\ref{eq:gen-fd}, to obtain
%
\begin{equation}%{{{
\hat \phi(k,l) = \hat G(k,l) \, \hat g(k,l),\quad 0<k,l<n-1
\end{equation}%}}}
%
with for a unit mesh
%
\begin{equation}%{{{
\begin{split}
\hat G(k,l) = \Bigg[
& 2a \left( \cos \frac{2\pi k}{n} - 1 \right) +
ib \sin \frac{2\pi k}{n} + c \,+ \\
& 2d \left( \cos \frac{2\pi l}{n} - 1 \right) +
ie \sin \frac{2\pi l}{n} + f
\Bigg]^{-1}
\end{split}
\end{equation}%}}}
%
To solve the Poisson equation, discretized as Eq.~\ref{eq:discrete-poisson}, we 
have $a=d=1$ and $b=c=e=f=0$ so we can simplify $\hat G(k,l)$ as
%
\begin{equation}%{{{
\hat G(k,l) = \frac{1}{2}\left[
\cos \frac{2\pi k}{n} +
\cos \frac{2\pi l}{n} -
2 \right]^{-1}
\end{equation}%}}}
%
The steps to compute the electric field can be summarized as follows:
%
\begin{center}%{{{
\begin{tikzpicture}[>=latex,thick]
	\matrix (m) [
		matrix of nodes,
		column sep=20mm,
		nodes={
			%line width=1pt,
			anchor=center,
			text centered,
			%minimum width=1cm,
			minimum height=8mm,
		},
%		txt/.style={text width=1.5cm,anchor=center},
	]
	{
		$\rho$ & $\hat \rho$ & $\hat \phi$ & $\phi$ & $\E$\\
	};
	\foreach \i [evaluate={\j=int(\i+1)}] in {1,...,4}{
		\draw[->] (m-1-\i) -- (m-1-\j);
	}
	\draw[draw=none] (m-1-1) -- (m-1-2) node[midway,above] {FFT};
	\draw[draw=none] (m-1-2) -- (m-1-3) node[midway,above] {$\hat G$};
	\draw[draw=none] (m-1-3) -- (m-1-4) node[midway,above] {IFFT};
	\draw[draw=none] (m-1-4) -- (m-1-5) node[midway,above] 
	{Eq.~\ref{eq:phi-to-E}};
\end{tikzpicture}
\end{center}%}}}
%
\begin{enumerate}
\item Compute the complex FFT $\hat g$ of $g$
\item Multiply each element of $\hat g$ by the corresponding complex coefficient 
$\hat G$, to obtain $\hat \phi$
\item Compute the inverse FFT of $\hat \phi$ to get $\phi$
\end{enumerate}
%
The complexity in the worst case is in $O(N_g \log_2 N_g)$ with the number of 
total points in the grid $N_g$.

\section{Force interpolation}

The force acting on a particle is computed similarly as the charge deposition, 
but in the reverse order. The electric field $\E$ is interpolated on the 
particle position, using the same interpolation function $W$. Then the force is 
computed as
\begin{equation}%{{{
\F_p = q \sum_{g} W(\x_g - \x) \E_g
\end{equation}%}}}


\section{Equations of motion}

