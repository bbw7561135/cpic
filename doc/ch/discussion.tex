\chapter{Discussion}
\label{ch:discussion}


\section{Conclusions}

The presented simulator faces challenging computational patterns that are 
representative in real case scenarios, which were the main aim of this work. 

Firstly, when using dependencies in the neighbour chunks a chain of dependencies 
disallows the parallel execution and has been solved by the use of two different 
techniques: the coloring of the chunks and the \texttt{commutative} directive.

The model of communications used with MPI, consisting in the use of probe and 
receive can be proposed to the inclusion into TAMPI, as the current solution 
requires minor modifications.

The use of TAMPI leads to a more efficient execution and improved performance, 
as the runtime can fully optimize the time by doing other tasks, and avoiding 
the waiting time between MPI calls.

Using the model provided by OmpSs-2 based on tasks, the parallelization of the 
different stages of the simulation was possible. The only step which continues 
to be sequential is the solver, for which a proposed solution is described for 
the future work.

\section{Future work}

The main problem to be solved in the simulator is to address the scalability 
issues presented by the FFT, as the mitigations tested don't provide a good 
solution. One possibility is the interoperability of the OmpSs-2 runtime, 
nanos6, with external MPI processes with an additional mechanism of 
synchronization. In this way, the simulator can be fully parallelized, even at 
the core level. A step by step scheme for a configuration with $C$ CPUs 
available per node and $N$ nodes, is outlined as follows:
%
\begin{enumerate}
\item Begin the simulation as usual creating $P=N$ master processes, each with 
at least $N_c \ge 2C$ plasma chunks, to exploit the local parallelism of the $C$ 
CPUs.
\item Place the fields $\rho$, $\phi$ and $\E$ in a shared memory region, 
accessible by other child processes.
\item Create $K$ MPI child processes in each master process, with access to the 
shared memory and let them wait on a condition variable or the reception of a 
MPI message.  Ensure the number of points $N_g$ in the vertical dimension is 
divisible by $KP$.
\item Continue the simulation until it reaches the solver stage.
\item Ensure all tasks are finished, and wake all the child processes and then 
wait for them to finish.
\item In each child process execute the distributed FFTW with $KP$ processes, 
and use the shared memory to access the fields.
\item Once the FFT finishes, signal the master and put each child process to 
sleep again, waiting for a signal.
\item In the master process, the $\phi$ field is now ready in the shared memory 
region. If the simulation is not finished, go to step 4.
\end{enumerate}
%
The key concept is that we are moving temporally the threads of the OmpSs-2 
runtime away from the CPUs to let the MPI processes of the FFTW take control of 
the full parallelism using all the available CPUs. No change is needed in the 
FFTW library, and this method may benefit other programs with similar issues.

On the other hand, the physical results must be validated with a direct 
comparison with other simulators, as is very easy simulate non-realistic 
behavior without noticing. The different validation techniques provide some 
ground that the simulation follows the expected behavior, but don't guarantee 
any correctness.

Additionally, there are a large list of improvements that were planned and may 
be tested in a future work:

\begin{itemize}
\item Introduce more than 2 dimensions.
\item Fully electromagnetic simulation.
\item Relativistic particle movement.
\item Heterogeneous architecture (GPU+CPU).
\item Better energy conserving codes.
\item Test other interpolation methods (reduce noise at computational cost).
\item Replace simulation units, so we avoid factor multiplications.
\item Visualization of big simulations (paraview).
\item Introduction of probe+receive operations in TAMPI.
\end{itemize}
