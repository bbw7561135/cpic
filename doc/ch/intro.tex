\chapter{Introduction}
\label{ch:intro}

It may be surprising to find out that the most common state of matter is plasma, 
when we look at the universe. The simulation of plasma has been increasingly 
researched since the computers began gaining computation speed, as it is quite 
complex and expensive to study in a physics laboratory. The particle-in-cell 
methods are now widely used for the simulation of different plasma phenomena, as 
they provide a good parallelization that can be exploited with today 
supercomputers.

A second big revolution began with the race for nuclear fusion, where 
lightweight atoms are heated until they are fused together into bigger atoms and 
subatomic particles, with the subsequent release of large amounts of energy.  
Currently diverse research groups focus on the confinement of plasma in order to 
heat it efficiently, so that more energy can be produced that the used to heat 
the plasma.

The turbulences and instabilities that occur in the plasma in magnetic 
confinement are one of the points of interest to be studied by the use of 
computer simulation, in order to obtain a successful geometry and configuration 
to avoid the loss of energy.

\section{Motivation}

The design and implementation of a plasma simulator is a useful way to find out 
the patterns and complexities of a parallel application used in the real world.  
Most of the existing PIC codes are highly tied to solve a specific set of 
simulations to work on some experiments and the documentation is usually not 
available or poor and the designs are the result of years added features without 
a clear design, which hardens the task of improving the performance.

The introduction of a programming model based on tasks and dependencies, as is 
the case of OmpSs-2, can free the programmer from the complexities of concurrent 
applications. But by building a complete simulation program, the difficulties 
and problems can be observed and addressed, benefiting other users in the 
future. At the same time, the structure and decisions are completely documented 
to facilitate the solution of similar problems in the future.

\section{Objectives}

One of the main objectives of the simulation is the use of the data-flow 
execution model provided by OmpSs-2 to find the challenging computational 
patterns that occur in a complete and real application, and to exploit at 
maximum the available parallelism provided by the HPC clusters.
%
Furthermore the Task Aware MPI library (TAMPI), will be compared against MPI to 
measure the performance in communications in a complex simulation scenario.

The challenges found during the design of the simulator will be used to improve 
the current solutions provided by the programming model and propose new 
alternatives.

\section{Structure}
%
The structure of the document follows the diagram shown in the 
figure~\ref{fig:structure}. In the chapter~\ref{ch:plasma-sim}, plasma is 
described as a physical phenomenon and we focus on the relevant properties that 
we want to study, from which we derive a mathematical model.  The discretization 
of the model allows the computer simulation by using numerical algorithms, and 
is discussed in the chapter~\ref{ch:discrete-model}. A sequential prototype is 
designed to test the proposed model in chapter~\ref{ch:sequential}.  Then, 
following the techniques described in the chapter~\ref{ch:techniques} a parallel 
simulator is build in chapters~\ref{ch:parallel-simulator} and~\ref{ch:comm}.
Finally, the performance of the simulator is addressed in the 
chapter~\ref{ch:analysis}, leading to the conclusions and future work in the 
chapter~\ref{ch:discussion}.
%
\begin{figure}[ht]%{{{
\centering
\scalebox{0.7} {
\begin{tikzpicture}[>=latex,thick]
	\matrix (m) [
		matrix of nodes,
		column sep=5mm,
		row sep=5mm,
		nodes={
			draw, % General options for all nodes
			line width=1pt,
			anchor=center,
			text centered,
			rounded corners,
			minimum width=5cm,
			minimum height=8mm,
		},
		txt/.style={text width=1.5cm,anchor=center},
	]
	{
		Physical phenomenon \\
		Mathematical model \\
		Discretization \\
		Numerical algorithms \\
		|[fill=black!10]| Parallelization \\
		Simulation program \\
		Computer experiment \\
	};
	\foreach \i [evaluate={\j=int(\i+1)}] in {1,...,6}{
		\draw[->] (m-\i-1) -- (m-\j-1);
	}
	\draw [
		decorate,decoration={brace,amplitude=5pt,raise=10pt},
	] (m-1-1.north east) -- (m-2-1.south east) node 
	[black,midway,right,xshift=18pt] {Chapter~\ref{ch:plasma-sim}};
	\draw [
		decorate,decoration={brace,amplitude=5pt,raise=10pt},
	] (m-3-1.north east) -- (m-4-1.south east) node 
	[black,midway,right,xshift=18pt] {Chapter~\ref{ch:discrete-model}};
	\draw [
		decorate,decoration={brace,amplitude=5pt,raise=10pt},
	] (m-5-1.north east) -- (m-5-1.south east) node 
	[black,midway,right,xshift=18pt]
		{Chapters~\ref{ch:parallel-simulator} and~\ref{ch:comm}};
\end{tikzpicture}
}
\caption{Principal steps in a computer simulation experiment: This project 
focuses on the parallelization step.}
\label{fig:structure}
\end{figure}%}}}
%


\section{Related work}

The simulation of plasma began with the first simulations in the 1950s with the
John Dawson codes for 1D simulation. In 1965 Hockney and Buneman introduced the
direct Poisson solver, which allowed the first useful electrostatic simulations.
In the 1970s, the theory of electrostatic PIC was developed by Langdon, leading
to the first electromagnetic codes.

Finally, from 1980 to the 90s the two main bibles of particle-in-cell codes were
produced  by B. Langdon and C. Birdsall in 1975 \cite{birdsall} and by Hockney
and Eastwood in 1988 \cite{hockney}.

At the Plasma Theory and Simulation Group of the University of California,
Berkeley the XOOPIC \cite{xoopic} family of well known codes were released in
the 1990s. There are a lot of specific PIC codes which are currently used for 
the simulation of various phenomena, mostly centered in fusion reactors: 
ELMFIRE, GENE, GTC, ORB5, PAR-T and EUTERPE \cite{euterpe}.

