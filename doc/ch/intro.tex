\chapter{Introduction}
\label{ch:intro}

It may be surprising to find out that the most common state of matter is plasma 
when we look at the universe. A plasma is an ionized gas consisting of ions and
free electrons distributed over a region in space.
in which at least one electron 
of the atom is separated, so it remains positively charged (ionized) 
\cite{chen}.  Usually this happens in the vacuum

\section{Motivation}

% Why?

\section{Objectives}

\section{Context}

\section{Structure}
%
\begin{figure}[h]
%\begin{wrapfigure}{O}{0.3\textwidth}
\centering
\scalebox{0.7} {
\begin{tikzpicture}[>=latex,thick]
	\matrix (m) [
		matrix of nodes,
		column sep=5mm,
		row sep=5mm,
		nodes={
			draw, % General options for all nodes
			line width=1pt,
			anchor=center,
			text centered,
			rounded corners,
			minimum width=5cm,
			minimum height=8mm,
		},
		txt/.style={text width=1.5cm,anchor=center},
	]
	{
		Physical phenomenon \\
		Mathematical model \\
		Discretization \\
		Numerical algorithms \\
		|[fill=black!10]| Parallelization \\
		Simulation program \\
		Computer experiment \\
	};
	\foreach \i [evaluate={\j=int(\i+1)}] in {1,...,6}{
		\draw[->] (m-\i-1) -- (m-\j-1);
	}
	\draw [
		decorate,decoration={brace,amplitude=5pt,raise=10pt},
	] (m-1-1.north east) -- (m-1-1.south east) node 
	[black,midway,right,xshift=18pt] {Chapter~\ref{ch:plasma-intro}};
	\draw [
		decorate,decoration={brace,amplitude=5pt,raise=10pt},
	] (m-2-1.north east) -- (m-2-1.south east) node 
	[black,midway,right,xshift=18pt] {Chapter~\ref{ch:plasma-sim}};
	\draw [
		decorate,decoration={brace,amplitude=5pt,raise=10pt},
	] (m-3-1.north east) -- (m-4-1.south east) node 
	[black,midway,right,xshift=18pt] {Chapter~\ref{ch:discrete-model}};

\end{tikzpicture}
}
\caption{Principal steps in computer experiment}
\label{fig:structure}
\end{figure}
%\end{wrapfigure}
%
The structure of the document follows the diagram shown in the 
figure~\ref{fig:structure}. In chapter~\ref{ch:intro}, plasma is described as a 
physical phenomenon and we focus on the relevant properties that we want to 
study, from which we derive a mathematical model.  In 
chapter~\ref{ch:plasma-sim} and~\ref{sec:test}. The discretization of the 
mathematical model allows the computer simulation by using numerical algorithms.
