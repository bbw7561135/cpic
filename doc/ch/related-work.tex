\chapter{Related work}

The simulation of plasma began with the first simulations in the 1950s with the
John Dawson codes for 1D simulation. In 1965 Hockney and Buneman introduced the
direct Poisson solver, which allowed the first useful electrostatic simulations.
In the 1970s, the theory of electrostatic PIC was developed by Langdon, leading
to the first electromagnetic codes.

Finally, from 1980 to the 90s the two main bibles of particle-in-cell codes were
produced  by B. Langdon and C. Birdsall in 1975 \cite{birdsall} and by Hockney
and Eastwood in 1988 \cite{hockney}.

At the Plasma Theory and Simulation Group of the University of California,
Berkeley the XOOPIC \cite{xoopic} family of well known codes were released in
the 1990s.

The are a lot of specific PIC codes which are currently used for the simulation
of various phenomena, mostly centered in fusion reactors: ELMFIRE, GENE, GTC,
ORB5, PAR-T and EUTERPE \cite{euterpe}.

