\chapter{Plasma simulation}
\label{ch:plasma-sim}

% Write about how plasma can be simulated with a computer. The methods regarding 
% *only* on numerical methods to simulate plasma, not the specific to HPC ones.


\section{Everyday plasmas}

It may be surprising to find out that when we look at the universe, the most
common state of matter is plasma, which is a ionized gas formed by free
electrons and ions at a region in space--often known as the fourth state of
matter.

Most common forms of plasma only occur in vacuum, as otherwise the air cools the
plasma and returns to a gas. However, is quite common to find plasma in the 
space: The sun, our closest star, is a giant ball of plasma.

In our planet, we can see forms of plasma almost every day. A storm day the
lightnings. The spark a piezoelectric lighters, which is the very same
principle that occurs in gasoline engines, and in the lightning of a storm.
The Aurora Borealis, of the lightning of a fluorescent tube are common examples.

A precise definition of a plasma is given by Chen~\cite{chen} as \textit{``a
quasineutral gas of charged and neutral particles which exhibits collective
behavior''}.


\section{The particle-in-cell method}

The plasmas we are interested in study are modeled by the Vlasov equation, which 
describes the time evolution of a distribution function of plasma $f$.

\begin{equation}
{\frac {\partial f}{\partial t}}+{\frac {\operatorname {d} \mathbf {r} 
}{\operatorname {d} t}}\cdot {\frac {\partial f}{\partial \mathbf {r} }}+{\frac 
{\operatorname {d} \mathbf {p} }{\operatorname {d} t}}\cdot {\frac {\partial 
f}{\partial \mathbf {p} }}=0
\end{equation}

%TODO: Show the main equation
Solving the Vaslov equation requires a large amount of numerical resources. The 
particle in cell method, approximates the solution by discretization of the 
fields and by interpolation of the grid to the particles. The method is divided 
in four main phases
%
\begin{itemize}
\item \textbf{Charge accumulation}: The charge density is interpolated in the 
grid from the particle positions.
\item \textbf{Solve field equation}: From the charge density $\rho$ the electric 
potential is obtained $\phi$ and then the electric field $\E$.
\item \textbf{Interpolation of electric field}: The electric field is  
interpolated back to the particle positions.
\item \textbf{Particle motion}: The force is computed from the electric field at 
the particle position and the particle is moved accordingly.
\end{itemize}
%
%\todo[inline]{Complete the description of the method}


%\section{1D electrostatic simulation}
%The magnetic field is ignored.
%
%\section{2D simulation}
%The magnetic field is not ignored.
%
%\section{Electromagnetism}
%
%\subsection{Background magnetic field}
%
%To introduce the magnetic field, the equations are:
%
%$$ $$


In order to move the particles, the equations of motion need to be solved:
%
\begin{equation}
m \frac{d\v}{dt} = q (\E + \v \times \B)
\end{equation}
\begin{equation}
\frac{d\v}{dt}=\v
\end{equation}
%
The charge density $\rho$ is a scalar field which describes the charge 
accumulated by the particles, and needs to be updated when the particles move.
%
Once we have the charge density $\rho$ we can compute the electric field $\E$ by 
the integration of the field equations
%
\begin{equation}
\E = -\nabla \phi
\end{equation}
\begin{equation}
\nabla \cdot \E = \frac{\rho}{\epsilon_0}
\end{equation}
%
Which can be combined into the Poisson equation
%
\begin{equation}
\label{eq:poisson}
\nabla^2\phi = - \frac{\rho}{\epsilon_0}
\end{equation}
%
Different methods can be used to obtain the electric field, but we will focus on 
matrix and spectral methods.
