\chapter{Plasma simulation}
\label{ch:plasma-sim}

% Write about how plasma can be simulated with a computer. The methods regarding 
% *only* on numerical methods to simulate plasma, not the specific to HPC ones.

\section{The particle-in-cell method}

%TODO: Show the main equation
Solving the Vaslov equation requires a large amount of numerical resources. The 
particle in cell method, approximates the solution by discretization of the 
fields and by interpolation of the grid to the particles. The method is divided 
in four main phases
%
\begin{itemize}
\item \textbf{Charge accumulation}: The charge density is interpolated in the 
grid from the particle positions.
\item \textbf{Solve field equation}: From the charge density $\rho$ the electric 
potential is obtained $\phi$ and then the electric field $\E$.
\item \textbf{Interpolation of electric field}: The electric field is  
interpolated back to the particle positions.
\item \textbf{Particle motion}: The force is computed from the electric field at 
the particle position and the particle is moved accordingly.
\end{itemize}
%
\todo[inline]{Complete the description of the method}


%\section{1D electrostatic simulation}
%The magnetic field is ignored.
%
%\section{2D simulation}
%The magnetic field is not ignored.
%
%\section{Electromagnetism}
%
%\subsection{Background magnetic field}
%
%To introduce the magnetic field, the equations are:
%
%$$ $$

\section{Particle mover}

In order to move the particles, the equations of motion need to be solved:
%
\begin{equation}
m \frac{d\v}{dt} = q (\E + \v \times \B)
\end{equation}
\begin{equation}
\frac{d\v}{dt}=\v
\end{equation}
%
Several methods are available, but we will focus on the Boris integrator.

\section{Charge accumulation}

The charge density $\rho$ is a scalar field

\section{Field equations}

Once we have the charge density $\rho$ we can compute the electric field $\E$ by 
the integration of the field equations
%
\begin{equation}
\E = -\nabla \phi
\end{equation}
\begin{equation}
\nabla \cdot \E = \frac{\rho}{\epsilon_0}
\end{equation}
%
Which can be combined into the Poisson equation
%
\begin{equation}
\label{eq:poisson}
\nabla^2\phi = - \frac{\rho}{\epsilon_0}
\end{equation}
%
Different methods can be used to obtain the electric field, but we will focus on 
matrix and spectral methods.

