\chapter{Plasma simulation}
\label{ch:plasma-sim}

% Write about how plasma can be simulated with a computer. The methods regarding 
% *only* on numerical methods to simulate plasma, not the specific to HPC ones.

\section{The particle-in-cell method}

%TODO: Show the main equation
Solving the Vaslov equation requires a large amount of numerical resources. The 
particle in cell method, approximates the solution by discretization of the 
fields.

The method is divided in four main phases: 

\begin{itemize}
\item Particle motion.
\item Charge accumulation.
\item Solve field equation.
\item Interpolation of fields in particle position.
\end{itemize}


%\section{1D electrostatic simulation}
%The magnetic field is ignored.
%
%\section{2D simulation}
%The magnetic field is not ignored.
%
%\section{Electromagnetism}
%
%\subsection{Background magnetic field}
%
%To introduce the magnetic field, the equations are:
%
%$$ $$

\section{Particle mover}

In order to move the particles, the equations of motion need to be solved:
%
\begin{equation}
m \frac{d\v}{dt} = q (\E + \v \times \B)
\end{equation}
\begin{equation}
\frac{d\v}{dt}=\v
\end{equation}
%
Several methods are available, but we will focus on the Boris integrator.

\subsection{Boris integrator}

Consists of three steps:
%
\begin{enumerate}
\item Add half of the electric impulse
\item Rotate
\item Add the remaining half electric impulse
\end{enumerate}
%
The Boris integrator computes the velocity of a particle in a constant electric 
field $\E$ and a constant magnetic field $\B$. We have the velocity 
$\v_{t-\Delta t/2}$ of the particle at $t-\Delta t/2$ as we use the leapfrog 
integrator.
%
\begin{figure}[h]
\centering
\begin{tikzpicture}[
	scale=2,
	>=latex]

	\def\centerarc[#1](#2)(#3:#4:#5)% Syntax: [draw options] (center) (initial angle:final angle:radius)
		{\draw[#1] ($(#2)+({#5*cos(#3)},{#5*sin(#3)})$) arc (#3:#4:#5); }

	\def\startangle{-25}
	\def\midangle{0}
	\def\endangle{25}
	\def\radius{2.0}
	\pgfmathsetmacro{\vlen}{\radius*tan(\startangle)}%

	\coordinate (O) at (0,0);
	\coordinate (S) at (\startangle:\radius);
	\coordinate (E) at (\endangle:\radius);

%	\centerarc[dashed](O)(\startangle:\endangle:\radius);
	\centerarc[->](O)(\startangle:\endangle:0.2*\radius);

	\draw (O)+(0.4,0.1) node [right] {$\theta$};

	\draw [thick,->] (O) -- (E) node [midway, above] {$\V{v^+}$};
	\draw [thick,->] (O) -- (S) node [midway, below] {$\V{v^-}$};

	\path (S) +(\startangle-90:\vlen) coordinate (V1E);
%	\path (E) +(\endangle-90:\vlen) coordinate (V3E);

%	%\draw [->] (E) -- (V3E);
	\draw [->] (S) -- (V1E) node [midway, right] {$\V{v'} \times \V t$};
%
	\draw [->] (O) -- (V1E) node [midway, above] {$\V{v'}$};
	\draw [->] (S) -- (E) node [near end, right] {$\V{v'} \times \V s$};

%	\draw [fill=white] (O) circle (0.02);

\end{tikzpicture}
\caption{Velocity space rotation from $\v-$ to $\v+$}
\end{figure}
%
\paragraph{Add half electric impulse} We define $\V{v^-}$ as the velocity after 
half a electric impulse:
$$\v^- = \v_{t-\dt/2} + \frac{q \E}{m} \frac{\dt}{2}$$

\paragraph{Rotate for the magnetic field} The rotation is done in two steps, 
first the half rotation is computed, with an angle of $\theta/2$:
$$\v' = \v^- + \v^- \times \V t $$

Then the rotation is completed by symmetry, using the $\V s$ vector
$$ \V s = \frac{2 \V t}{1 + \V t^2} $$
as
$$ \V{v^+} = \V{v^-} + \V{v}' \times \V{s} $$

\section{Charge accumulation}

The charge density $\rho$ is a scalar field

\section{Field equations}

Once we have the charge density $\rho$ we can compute the electric field $\E$ by 
the integration of the field equations
%
\begin{equation}
\E = -\nabla \phi
\end{equation}
\begin{equation}
\nabla \cdot \E = \frac{\rho}{\epsilon_0}
\end{equation}
%
Which can be combined into the Poisson equation
%
\begin{equation}
\nabla^2\phi = - \frac{\rho}{\epsilon_0}
\end{equation}
%
Different methods can be used to obtain the electric field, but we will focus on 
finite difference equations and spectral methods.

\subsection{Finite differences}

\begin{align}
\E_x(x,y) &= \frac{\phi(x-1,y) - \phi(x+1,y)}{2\,\Delta x} \\
\E_y(x,y) &= \frac{\phi(x,y-1) - \phi(x,y+1)}{2\,\Delta y}
\end{align}

\subsection{Spectral methods}


